% Test change for git
%===========================================
% Preamble
\documentclass[11pt]{article}

%**********
%Dependencies
\usepackage[left]{lineno}
\usepackage{titlesec}
\usepackage{amsmath}
\usepackage{amsfonts}
\usepackage{amssymb}
%\usepackage[utf8]{inputenc}
\usepackage{color,soul}
%\usepackage{setspace}
%\usepackage{times}
\usepackage{ogonek}


% Packages from Am. Nat. Template
\usepackage[sc]{mathpazo} %Like Palatino with extensive math support
\usepackage{fullpage}
\usepackage[authoryear,sectionbib,sort]{natbib}
\linespread{1.15}
\usepackage[utf8]{inputenc}
\usepackage{lineno}

% Change default margins
\usepackage[top=1in, bottom=1in, left=1in, right=1in]{geometry}


% Change subsection numbering
\renewcommand\thesubsection{\arabic{subsection})}
\renewcommand\thesubsubsection{}

% Subsubsection Title Formatting
\titleformat{\subsubsection}    
{\normalfont\fontsize{12pt}{17}\itshape}{\thesubsubsection}{12pt}{}

% Equation numbering
\newcommand\numberthis{\addtocounter{equation}{1}\tag{\theequation}}

% Definitions
\def\mathbi#1{\textbf{\em #1}}


%===========================================

\begin{document}
\title{Grabbing the ruby in the rubbish: Adaptive hypotheses for the evolution of suppressed recombination}
\author{Colin Olito}
\date{8 October 2018}
\maketitle

\textbf{Last updated:} \date{\today}

\section*{Notes}

This started with a conversation I had with Tim around the time we finished the X-chromosome inversion paper \citep{ConnallonOlito2018}. We spoke briefly about how one of the things we talk about in that paper, where the invasion of inversions depends on the likelihood of an inversion capturing the two SA loci being free of deleterious mutations, was very similar to Allen Orr's simple models of adaptation in asexual populations \citep{Orr2000}. In both cases, whether the invasion (or asexual genome) successfully invades depends mostly on it being one of the 'lucky few' in the '$0$' class for the number of  deleterious mutations. One of the interesting findings from Orr's model was that the rate of adaptation in asexuals highest when the genome-wide mutation rate was roughly equal to the strength of selection against deleterious mutations ($U \approx s_d$). The potentially cool idea we had was that it should be possible to modify Orr's model to make predictions about the optimal size of inversions. As the inversion size increases, it is more likely to capture beneficial mutations, but it is also more likely to capture a deleterious mutation that totally negates the benefits of the beneficial ones. Also, because the equilibirum frequency of delterious mutations should be different on the X vs.~Autosomes, we thought it could be cool to see how the optimal inversion sizes differ between these two genomic regions. Intuitively, X-linked inversions should be bigger because the standing load of deleterious mutations is lower, and therefore inversions capturing no deleterious mutations should be larger on average. 
\bigskip

Since that conversation, Mark Kirkpatrick published a paper in Molecular Ecology \citep{ChangKirkpatrick2018} with a cool analysis of \textit{Drosophila} and mosquito data which seems to support this prediction. They also presented a very crappy theoretical model (a slight modification of Charlesworth et al.~(1987) that sorta-kinda supported their finding, but was pretty weak beer.
\bigskip



\subsection{SexGen meetings, and expanding the idea}
		
	Since coming to Lund, I've been going to our weekly SexGen meetings with Jessica and Bengt Hansson's labs. Some of the work in Bengt's group has got me thinking a lot about the evolution of sex chromosomes, and in particular the evolution of recombination suppression on proto sex chromosomes. In particular, I've come to have a better appreciation for how little we know about how and why we so commonly see the evolution of recombination suppression between sex chromosomes. I have also been really interested in following up on this idea with Tim about extending \citet{Orr2000} to make predictions about optimal inversion sizes. This got me thinking \ldots
	\bigskip

	I think it would be relatively easy to generalize the question about optimal inversion sizes to address the evolution of recombination suppression on proto sex chromosomes. A couple key differences: (i) we would be interested adaptation on proto sex-chromosomes (i.e., with different transmission patterns, but still with large PAR's where recombination takes place). I\citet{OrrKim1998} actually develop equations for this scenario. (ii) Instead of inversions capturing beneficial vs.~deleterious mutations, we would be interested in regions of suppressed recombination that capture sexually antagonistic alleles favouring male fitness (on the proto-Y) or female fitness (on the proto-X) vs.~deleterious mutations. In this case, the equilibrium frequencies of SA alleles will be important, and these will depend on the sexually antagonistic selection and dominance coefficients. In both cases, The math should be relatively straightforward.
	\bigskip

	In the end, the idea , I think this idea boils down to a slightly different use of the Poisson distribution. In \citet{Orr2000} and \citet{OrrKim1998}, the probability of deleterious mutations occuring on a non-recombining asexual genome, or genome region, is modeled as a poisson process with 

\begin{equation*}
	\Pr(k~\text{events}) = e^{-\lambda} \frac{\lambda^{k}}{k!}.
\end{equation*}

	\noindent In the present case, rather than beinga able to use this to calculate the probability of an asexual genome or non-recombining chromosome has zero deleterious mutations, we are interested in the probability that non-recombining segments (e.g., inversions) of different sizes capture no deleterious mutations. Letting $U_d$ equal the per-chromosome deleterius  mutation rate, and$L$ equal the fraction of the total chromosome length that the inversion or non-recombining region spans, we can now write  

\begin{equation*}
	\Pr(k~\text{mutations}~\text{captured~by~inversion~of~size~}L) = e^{-U_d L} \frac{(U_d L)^{k}}{k!}.
\end{equation*}

	\noindent From \citet{Orr2000,OrrKim1998} we know that the maximal rate of adaptive substitution will occur when $U_d = s_d$. Using our formulation for inversions of different sizes, the rate of adaptation will be a function of both the per-chromosome delterious mutation rate and the size of the inversion. Also, it's not clear to me right away how things will play out with both beneficial and delterious mutations. Obviously, larger inversions will be more likely to capture more beneficial as well as deleterious mutations... but I don't know right now whether this is likely to just 'drop out' of the maximizing condition. We'll see.


%%%%%%%%%%%%%%%%%%%%%%%%%%%%%%%%%%%%%
%%%%%%%%%%%%%%%%%%%%%%%%%%%%%%%%%%%%%
\section*{Some simple models}

Consider a scenario of adaptation in a diploid population of size $N$, in which adaptation occurs by the substitution of new beneficial mutations. We assume that new mutations arise at a per-chromosome rate of $U$, and that a fixed proportion $p_b$ are beneficial while the remainder are deleterious $p_d = 1 - p_b$. Hence, the rate that new beneficial and deleterious mutations occur is $U_b = U p_b$ and $U_d = U p_d$. The fitness effects of new beneficial mutations follow a distribution, $f(s_b)$, with the average mutation enjoying a selective advantage of $\overline{s}_b$. On an autosome, the average new beneficial mutation has a probability of fixation of approximately $2\overline{s}_b$ (under sufficiently weak selection). Now, consider the establishment of a chromosomal inversion (or some other mechanism of recombination suppression) spanning a fraction $L$ of the total chromosome length that may capture beneficial mutations. The probability that such an inversion captures $n_b$ beneficial mutations will follow a Poisson process, where   

\begin{equation}
	P(n_b~|~L) = e^{-U_b L} \frac{(U_b L)^{n_b}}{n_b!}.
\end{equation}

\noindent For now focus on the simplest case where a new inversion captures only a single beneficial mutation, but will return to the case where $n_b > 1$. The rate of adaptive substitution for such an inversion will be equal to the probability that the inversion captures a single beneficial mutation multiplied by the probability of fixation: $4N U_b s_b (e^{-U_b L} U_b L)$. The probability of fixation for inversions capturing a single beneficial mutation on equivalently sized X-chromosomes is $3 N U_b s_b (e^{-U_b L} U_b L)$.

Clearly, new inversions may capture both beneficial and deleterious mutations, and the average probability of fixation should reflect this. Therefore, following \citep{OrrKim1998, Orr2000}, we re-write the probability of fixation for a new autosomal inversion capturing a single beneficial mutation as $4N U_b s_{\text{eff}} (e^{-U_b L} U_b L)$. At mutation-selection balance, the number of deleterious mutations captured by an autosomal inversion of size $L$ is also Poisson distributed: $P_{n_d} = e^{-U_d L/s_d}(U_d L/h_d s_d)^{n_d}/n_d!$. When the fitness effects of deleterious mutations are much stronger than those of beneficial mutations ($s_d \gg \overline{s}_b$), adaptive substitution of the beneficial mutation captured by our inversion is determined by the probability that the inversion of size $L$ captures no deleterious mutations, $P_0 = e^{-U_d L/s_d}$. The effective strength of selection favouring the inversion is now $s_{\text{eff}} \approx \overline{s}_b e^{-U_d L/s_d}$. Bringing together the above terms gives an approximate rate of adaptive substitution for autosomal inversions carrying a single beneficial mutation of

\begin{equation} \label{eq:maxRateAuto}
	k_A = 4N U_b e^{-U_d L/s_d} \overline{s}_b (e^{-U_b L} U_b L).
\end{equation}

\noindent We can follow the same logic for X-linked inversions. At mutation-selection balance, and under weak mutation, we can approximate the probability that an X-linked inversion of size $L$ carries $n_d$ deleterious mutations as $P^X_{n_d} = e^{- 3 U_d L/s_d(3 - 2 h_d)}(3 U_d L/s_d(3 - 2 h_d))^{n_d}/n_d!$. Neglecting cases where inversions one or more deleterious mutations, we have $s^X_{\text{eff}} \approx \overline{s}_b e^{- 3 U_d L/s_d(3 - 2 h_d)}$, and the approximate rate of adaptive substitution for X-linked inversions of size $L$ is

\begin{equation} \label{eq:maxRateX}
	k_X = 3 N U_b e^{- 3 U_d L/s_d(3 - 2 h)} \overline{s}_b (e^{-U_b L} U_b L).
\end{equation}

\noindent If we assume that the dominance and fitness effect of deleterious mutations are independent (i.e., $\text{E}[h_d s_d] = \text{E}[h_d]\text{E}[s_d]$), then the above results can be generalized to the case where deleterious mutations follow some distribution of fitness effects by replacing $h_d$ and $s_d$ with their harmonic means, $\langle h_d \rangle$ and $\langle s_d \rangle$, respectively. Similarly, because Eq(\ref{eq:maxRateAuto}) and Eq(\ref{eq:maxRateX}) are linear in $s_b$, we can replace $s_b$ with the arithmetic mean $\overline{s}_b$.

\begin{subequations}
\begin{align}
	\text{E}[k_A] &= 4N U_b e^{-U_d L/\langle s_d \rangle} \overline{s}_b (e^{-U_b L} U_b L). \\
	\text{E}[k_X] &= 3 N U_b e^{- 3 U_d L/\langle s_d \rangle(3 - 2 \langle h \rangle)} \overline{s}_b (e^{-U_b L} U_b L).
\end{align}
\end{subequations}



\subsection*{Proto-sex chromosomes}

The probability of fixation for inversions capturing a single beneficial mutation on equivalently sized X-chromosomes is $3 N U_b s_b (e^{-U_b L} U_b L)$, and on an equivalently sized proto-Y chromosomes, it is $N U_b s_b (e^{-U_b L} U_b L)$.
%\newpage

% \begin{table}[!ht]
% \caption{Offspring genotypic frequencies resulting from outcrossing}
% \label{Table:OffFreq}
% \centering
% \begin{tabular}{l c c } \hline
%  Frequency & Genotype \\
% \hline
% $F_{11}$ & $AAM_1M_1$  \\
% $F_{12}$ & $AAM_1M_2$  \\
% $F_{13}$ & $AaM_1M_1$  \\
% $F_{14}$ & $AaM_1M_2$  \\
% $F_{22}$ & $AAM_2M_2$  \\
% $F_{23}$ & $AaM_2M_1$  \\
% $F_{24}$ & $AaM_2M_2$  \\
% $F_{33}$ & $aaM_1M_1$  \\
% $F_{34}$ & $aaM_1M_2$  \\
% $F_{44}$ & $aaM_2M_2$  \\
% \hline
% \end{tabular}
% \bigskip{}
% \\
% \end{table}



	
\newpage
\begin{thebibliography}{}

\bibitem[{Chang and Kirkpatrick(2018)Chang and Kirkpatrick}]{ChangKirkpatrick2018}
Chang, C., and M.~Kirkpatrick. 2018.
\newblock Inversions are bigger on the X chromosome.
\newblock Mol.~Ecol. 2018:1--8.

\bibitem[{Charlesworth et.~al.(1987)Charlesworth et.~al.}]{CharlesworthBarton1987}
Charlesworth, B., J.~A. Coyne, and N. Barton. 1987.
\newblock The relative rates of evolution of sex chromosomes and autosomes.
\newblock American Naturalist 130:113--146.

\bibitem[{Connallon et al.(2018)Connallon et al.}]{ConnallonOlito2018}
Connallon, T., C.~Olito, L.~Dutoit, H.~Popali, F.~Ruzicka, and L. Yong. 2018.
\newblock Local adaptation and the evolution of inversions on sex chromosomes and autosomes.
\newblock Phil.~Trans.~Roy.~Soc.~B 373:20170423.

\bibitem[{Orr(2000)Orr}]{Orr2000}
Orr, H.~A. 2000.
\newblock The rate of adaptation in asexuals.
\newblock Genetics 155:961--968.

\bibitem[{Orr and Kim(1998)Orr and Kim}]{OrrKim1998}
Orr, H.~A. and Y.~Kim. 1998.
\newblock An adaptive hypothesis for the evolution of the Y chromosome.
\newblock Genetics 150:1693--1698.


\end{thebibliography}

\end{document}
